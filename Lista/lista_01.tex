\documentclass[a4paper,12pt]{article}
\usepackage{amsmath}
\usepackage{amssymb}
\usepackage{booktabs}
\usepackage{geometry}
\usepackage{graphicx}

\geometry{margin=1in}

\title{Lista Prática}
\author{}
\date{}

\begin{document}

\maketitle

\section*{1.}

A terra pode ser aproximada por uma esfera de raio $R = 6367.5$ km. Uma localização na superfície da Terra é tradicionalmente dada por sua latitude $\phi$ e sua longitude $\theta$, que correspondem, respectivamente, à distância angular do equador e do meridiano principal. As coordenadas em 3 dimensões são dadas pelo vetor $(x,y,z)$, onde:
\[
\begin{aligned}
x &= R \cos(\theta) \sin(\phi), \\
y &= R \cos(\theta) \cos(\phi), \\
z &= R \sin(\theta).
\end{aligned}
\]

\subsection*{a.}
Faça uma função que receba as coordenadas e retorne o vetor posição correspondente. Teste nas seguintes cidades:

\begin{table}[h!]
    \centering
    \begin{tabular}{|l|c|c|}
        \hline
        \textbf{Cidade} & \textbf{Latitude ($\phi$)} & \textbf{Longitude ($\theta$)} \\
        \hline
        São Paulo & $-23.5505^\circ$ & $-46.6333^\circ$ \\
        Nova York & $40.7128^\circ$ & $-74.0060^\circ$ \\
        Tóquio & $35.6895^\circ$ & $139.6917^\circ$ \\
        Sydney & $-33.8688^\circ$ & $151.2093^\circ$ \\
        Londres & $51.5074^\circ$ & $-0.1278^\circ$ \\
        \hline
    \end{tabular}
    \caption{Coordenadas das cidades.}
\end{table}

\subsection*{b.}
Calcule as distâncias euclidianas em 3D entre São Paulo e as outras cidades utilizando o vetor posição.

\subsection*{c.}
Faça uma função que calcule a distância na superfície da Terra entre dois pontos dadas suas coordenadas. Utilize a fórmula $d(a,b) = R \angle(a,b)$ e teste no mesmo caso do exercício anterior.

\section*{2.}

Considere os dados da tabela:

\subsection*{Dados de 8 Países:}

\begin{table}[h!]
    \centering
    \begin{tabular}{|l|c|c|c|c|}
        \hline
        \textbf{País} & \textbf{PIB (trilhões USD)} & \textbf{População (milhões)} & \textbf{Inflação (\%)} & \textbf{Desemprego (\%)} \\
        \hline
        Brasil & 2.1 & 213 & 3.2 & 11.2 \\
        EUA & 21.4 & 331 & 2.1 & 3.7 \\
        China & 14.3 & 1439 & 2.9 & 3.8 \\
        Alemanha & 3.8 & 83 & 1.4 & 3.2 \\
        Japão & 4.9 & 126 & 0.5 & 2.8 \\
        França & 2.6 & 67 & 1.8 & 8.1 \\
        Reino Unido & 2.8 & 67 & 2.5 & 4.1 \\
        Canadá & 1.7 & 38 & 1.9 & 5.7 \\
        \hline
    \end{tabular}
    \caption{Dados econômicos e sociais de 8 países.}
\end{table}

\subsection*{a.}
Compare a semelhança entre o Brasil e outros países utilizando a distância euclidiana entre os vetores de características (PIB, População, Inflação, Desemprego).

\subsection*{b.}
Faça a padronização dos dados utilizando a técnica de z-score em cada uma das características. Após a padronização, compare novamente a semelhança entre os países. Explique a diferença nos resultados comparados com o item anterior.

\subsection*{c.}
Compute a matriz de correlação entre os países.

\end{document}