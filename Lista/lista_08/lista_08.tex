\documentclass[a4paper,12pt]{article}
\usepackage{amsmath}
\usepackage{amssymb}
\usepackage{booktabs}
\usepackage{geometry}
\usepackage{graphicx}
\usepackage[portuguese]{babel}  

\geometry{margin=0.5in}

\title{Lista Teórica 8}
\author{}
\date{}

\begin{document}

\maketitle
\begin{enumerate}
\item Mostre que $\mathbf{v}$ é um autovetor de $A$ e determine o autovalor correspondente:
\begin{enumerate}
    \item[(a)] A = $\begin{bmatrix} 0 & 3 \\ 3 & 0 \end{bmatrix}$, $\mathbf{v} = \begin{bmatrix} 1 \\ 1 \end{bmatrix}$
    \item[(b)] A = $\begin{bmatrix} 1 & 2 \\ 2 & 1 \end{bmatrix}$, $\mathbf{v} = \begin{bmatrix} -3 \\ 3 \end{bmatrix}$
\end{enumerate}
\item Mostre que $\lambda$ é um autovalor de $A$ e determine o autovetor correspondente a esse autovalor:
\begin{enumerate}
    \item[(a)] A = $\begin{bmatrix} 2 & 2 \\ 2 & -1 \end{bmatrix}$, $\lambda = 3$
    \item[(b)] A = $\begin{bmatrix} 2 & 2 \\ 2 & -1 \end{bmatrix}$, $\lambda = -2$
\end{enumerate}
\item Dadas as matrizes:
$$A = \begin{bmatrix} 1 & 3 \\ -2 & 6\end{bmatrix},
B = \begin{bmatrix} 1 & 0 & 1 \\ 0 & 1 & 1 \\ 1 & 1 & 0 \end{bmatrix},
$$
calcule:
\begin{enumerate}
    \item[(a)] O polinômio característico de cada matriz.
    \item[(b)] Os autovalores de cada matriz.
    \item[(c)] A base para cada autoespaço.
    \item[(d)] A multiplicidade algébrica e geométrica de cada autovalor. 
\end{enumerate}
\item  Assuma que $A$ e $B$ são matrizes $n \times n$ com $\text{det}(A) = 3$ e $\text{det}(B) = -2$. Encontre:
\begin{enumerate}
    \item [(a)] $\text{det}(AB)$
    \item [(b)] $\text{det}(A^2)$
    \item [(c)] $\text{det}(B^{-1}A)$
    \item [(d)] $\text{det}(2A)$
    \item [(e)] $\text{det}(3B^T)$
    \item [(f)] $\text{det}(A A^T)$
\end{enumerate}
\item Mostre que os autovalores de uma matriz triangular superior da forma:
$$
A = \begin{bmatrix} a & b \\ 0 & d \end{bmatrix}
$$
são $\lambda_1 = a$ e $\lambda_2 = d$. e encontre os autoespacos correspondentes.
\end{enumerate}
\end{document}
